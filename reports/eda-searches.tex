\documentclass[]{article}
\usepackage{lmodern}
\usepackage{amssymb,amsmath}
\usepackage{ifxetex,ifluatex}
\usepackage{fixltx2e} % provides \textsubscript
\ifnum 0\ifxetex 1\fi\ifluatex 1\fi=0 % if pdftex
  \usepackage[T1]{fontenc}
  \usepackage[utf8]{inputenc}
\else % if luatex or xelatex
  \ifxetex
    \usepackage{mathspec}
  \else
    \usepackage{fontspec}
  \fi
  \defaultfontfeatures{Ligatures=TeX,Scale=MatchLowercase}
\fi
% use upquote if available, for straight quotes in verbatim environments
\IfFileExists{upquote.sty}{\usepackage{upquote}}{}
% use microtype if available
\IfFileExists{microtype.sty}{%
\usepackage{microtype}
\UseMicrotypeSet[protrusion]{basicmath} % disable protrusion for tt fonts
}{}
\usepackage[margin=1in]{geometry}
\usepackage{hyperref}
\hypersetup{unicode=true,
            pdftitle={Lab 2 - Checkpoint 4: Análise de dados de de projetos da Wikimedia},
            pdfborder={0 0 0},
            breaklinks=true}
\urlstyle{same}  % don't use monospace font for urls
\usepackage{color}
\usepackage{fancyvrb}
\newcommand{\VerbBar}{|}
\newcommand{\VERB}{\Verb[commandchars=\\\{\}]}
\DefineVerbatimEnvironment{Highlighting}{Verbatim}{commandchars=\\\{\}}
% Add ',fontsize=\small' for more characters per line
\usepackage{framed}
\definecolor{shadecolor}{RGB}{248,248,248}
\newenvironment{Shaded}{\begin{snugshade}}{\end{snugshade}}
\newcommand{\KeywordTok}[1]{\textcolor[rgb]{0.13,0.29,0.53}{\textbf{#1}}}
\newcommand{\DataTypeTok}[1]{\textcolor[rgb]{0.13,0.29,0.53}{#1}}
\newcommand{\DecValTok}[1]{\textcolor[rgb]{0.00,0.00,0.81}{#1}}
\newcommand{\BaseNTok}[1]{\textcolor[rgb]{0.00,0.00,0.81}{#1}}
\newcommand{\FloatTok}[1]{\textcolor[rgb]{0.00,0.00,0.81}{#1}}
\newcommand{\ConstantTok}[1]{\textcolor[rgb]{0.00,0.00,0.00}{#1}}
\newcommand{\CharTok}[1]{\textcolor[rgb]{0.31,0.60,0.02}{#1}}
\newcommand{\SpecialCharTok}[1]{\textcolor[rgb]{0.00,0.00,0.00}{#1}}
\newcommand{\StringTok}[1]{\textcolor[rgb]{0.31,0.60,0.02}{#1}}
\newcommand{\VerbatimStringTok}[1]{\textcolor[rgb]{0.31,0.60,0.02}{#1}}
\newcommand{\SpecialStringTok}[1]{\textcolor[rgb]{0.31,0.60,0.02}{#1}}
\newcommand{\ImportTok}[1]{#1}
\newcommand{\CommentTok}[1]{\textcolor[rgb]{0.56,0.35,0.01}{\textit{#1}}}
\newcommand{\DocumentationTok}[1]{\textcolor[rgb]{0.56,0.35,0.01}{\textbf{\textit{#1}}}}
\newcommand{\AnnotationTok}[1]{\textcolor[rgb]{0.56,0.35,0.01}{\textbf{\textit{#1}}}}
\newcommand{\CommentVarTok}[1]{\textcolor[rgb]{0.56,0.35,0.01}{\textbf{\textit{#1}}}}
\newcommand{\OtherTok}[1]{\textcolor[rgb]{0.56,0.35,0.01}{#1}}
\newcommand{\FunctionTok}[1]{\textcolor[rgb]{0.00,0.00,0.00}{#1}}
\newcommand{\VariableTok}[1]{\textcolor[rgb]{0.00,0.00,0.00}{#1}}
\newcommand{\ControlFlowTok}[1]{\textcolor[rgb]{0.13,0.29,0.53}{\textbf{#1}}}
\newcommand{\OperatorTok}[1]{\textcolor[rgb]{0.81,0.36,0.00}{\textbf{#1}}}
\newcommand{\BuiltInTok}[1]{#1}
\newcommand{\ExtensionTok}[1]{#1}
\newcommand{\PreprocessorTok}[1]{\textcolor[rgb]{0.56,0.35,0.01}{\textit{#1}}}
\newcommand{\AttributeTok}[1]{\textcolor[rgb]{0.77,0.63,0.00}{#1}}
\newcommand{\RegionMarkerTok}[1]{#1}
\newcommand{\InformationTok}[1]{\textcolor[rgb]{0.56,0.35,0.01}{\textbf{\textit{#1}}}}
\newcommand{\WarningTok}[1]{\textcolor[rgb]{0.56,0.35,0.01}{\textbf{\textit{#1}}}}
\newcommand{\AlertTok}[1]{\textcolor[rgb]{0.94,0.16,0.16}{#1}}
\newcommand{\ErrorTok}[1]{\textcolor[rgb]{0.64,0.00,0.00}{\textbf{#1}}}
\newcommand{\NormalTok}[1]{#1}
\usepackage{graphicx,grffile}
\makeatletter
\def\maxwidth{\ifdim\Gin@nat@width>\linewidth\linewidth\else\Gin@nat@width\fi}
\def\maxheight{\ifdim\Gin@nat@height>\textheight\textheight\else\Gin@nat@height\fi}
\makeatother
% Scale images if necessary, so that they will not overflow the page
% margins by default, and it is still possible to overwrite the defaults
% using explicit options in \includegraphics[width, height, ...]{}
\setkeys{Gin}{width=\maxwidth,height=\maxheight,keepaspectratio}
\IfFileExists{parskip.sty}{%
\usepackage{parskip}
}{% else
\setlength{\parindent}{0pt}
\setlength{\parskip}{6pt plus 2pt minus 1pt}
}
\setlength{\emergencystretch}{3em}  % prevent overfull lines
\providecommand{\tightlist}{%
  \setlength{\itemsep}{0pt}\setlength{\parskip}{0pt}}
\setcounter{secnumdepth}{0}
% Redefines (sub)paragraphs to behave more like sections
\ifx\paragraph\undefined\else
\let\oldparagraph\paragraph
\renewcommand{\paragraph}[1]{\oldparagraph{#1}\mbox{}}
\fi
\ifx\subparagraph\undefined\else
\let\oldsubparagraph\subparagraph
\renewcommand{\subparagraph}[1]{\oldsubparagraph{#1}\mbox{}}
\fi

%%% Use protect on footnotes to avoid problems with footnotes in titles
\let\rmarkdownfootnote\footnote%
\def\footnote{\protect\rmarkdownfootnote}

%%% Change title format to be more compact
\usepackage{titling}

% Create subtitle command for use in maketitle
\newcommand{\subtitle}[1]{
  \posttitle{
    \begin{center}\large#1\end{center}
    }
}

\setlength{\droptitle}{-2em}
  \title{Lab 2 - Checkpoint 4: Análise de dados de de projetos da Wikimedia}
  \pretitle{\vspace{\droptitle}\centering\huge}
  \posttitle{\par}
  \author{}
  \preauthor{}\postauthor{}
  \date{}
  \predate{}\postdate{}


\begin{document}
\maketitle

O objetivo deste estudo é investigar dados de pesquisas realizadas em
projetos da Wikimedia. Os dados utilizados foram coletados durante 8
dias pelo registro de dados de eventos (logs de eventos), e
disponibilizados pela Wikimedia Foundation. Os dados são usados para
avaliar a satisfação do usuário com os resultados das buscas.
Escolhidos aleatoriamente, os resultados são armazenados em logs,
permitindo identificar o tempo que os usuários permanecem nas páginas
visitadas. Esse problema de exploração de dados foi utilizado pela
Wikimedia Foundation em 2016 para recrutar pessoas.

\subsection{1. Configurações iniciais, conjunto de dados e variáveis
utilizadas}\label{configuraaaes-iniciais-conjunto-de-dados-e-variaveis-utilizadas}

O primeiro passo para iniciarmos o estudo é conhecer a base de dados e
as variávies utilizadas. Mas antes disso temos que fazer as
configurações iniciais. O código a seguir mostra as bibliotecas e
dependências utilizadas.

\begin{Shaded}
\begin{Highlighting}[]
\KeywordTok{library}\NormalTok{(tidyverse)}
\end{Highlighting}
\end{Shaded}

\begin{verbatim}
## -- Attaching packages ----------------------------------------------------- tidyverse 1.2.1 --
\end{verbatim}

\begin{verbatim}
## v ggplot2 2.2.1     v purrr   0.2.4
## v tibble  1.4.2     v dplyr   0.7.4
## v tidyr   0.8.0     v stringr 1.3.0
## v readr   1.1.1     v forcats 0.3.0
\end{verbatim}

\begin{verbatim}
## -- Conflicts -------------------------------------------------------- tidyverse_conflicts() --
## x dplyr::filter() masks stats::filter()
## x dplyr::lag()    masks stats::lag()
\end{verbatim}

\begin{Shaded}
\begin{Highlighting}[]
\KeywordTok{library}\NormalTok{(here)}
\end{Highlighting}
\end{Shaded}

\begin{verbatim}
## here() starts at C:/Users/Gleyser Guimarães/Documents/GitHub/lab2-cp4-Gleyser
\end{verbatim}

\begin{Shaded}
\begin{Highlighting}[]
\KeywordTok{library}\NormalTok{(lubridate)}
\end{Highlighting}
\end{Shaded}

\begin{verbatim}
## 
## Attaching package: 'lubridate'
\end{verbatim}

\begin{verbatim}
## The following object is masked from 'package:here':
## 
##     here
\end{verbatim}

\begin{verbatim}
## The following object is masked from 'package:base':
## 
##     date
\end{verbatim}

\begin{Shaded}
\begin{Highlighting}[]
\KeywordTok{library}\NormalTok{(ggplot2)}
\KeywordTok{theme_set}\NormalTok{(}\KeywordTok{theme_bw}\NormalTok{())}
\end{Highlighting}
\end{Shaded}

\begin{Shaded}
\begin{Highlighting}[]
\NormalTok{buscas =}\StringTok{ }\KeywordTok{read_csv}\NormalTok{(here}\OperatorTok{::}\KeywordTok{here}\NormalTok{(}\StringTok{"data/search_data.csv"}\NormalTok{)) }\OperatorTok
\StringTok{    }\KeywordTok{mutate}\NormalTok{(}\DataTypeTok{day=}\KeywordTok{round_date}\NormalTok{(session_start_date, }\DataTypeTok{unit =} \StringTok{"day"}\NormalTok{))}
\end{Highlighting}
\end{Shaded}

\begin{verbatim}
## Parsed with column specification:
## cols(
##   session_id = col_character(),
##   search_index = col_integer(),
##   session_start_timestamp = col_double(),
##   session_start_date = col_datetime(format = ""),
##   group = col_character(),
##   results = col_integer(),
##   num_clicks = col_integer(),
##   first_click = col_integer()
## )
\end{verbatim}

\begin{Shaded}
\begin{Highlighting}[]
\NormalTok{## Parsed with column specification:}
 \KeywordTok{cols}\NormalTok{(}
   \DataTypeTok{session_id =} \KeywordTok{col_character}\NormalTok{(),}
   \DataTypeTok{search_index =} \KeywordTok{col_integer}\NormalTok{(),}
   \DataTypeTok{session_start_timestamp =} \KeywordTok{col_double}\NormalTok{(),}
   \DataTypeTok{session_end_timestamp =} \KeywordTok{col_double}\NormalTok{(),}
   \DataTypeTok{session_start_date =} \KeywordTok{col_datetime}\NormalTok{(}\DataTypeTok{format =} \StringTok{""}\NormalTok{),}
   \DataTypeTok{session_end_date =} \KeywordTok{col_datetime}\NormalTok{(}\DataTypeTok{format =} \StringTok{""}\NormalTok{),}
   \DataTypeTok{checkin =} \KeywordTok{col_integer}\NormalTok{(),}
   \DataTypeTok{group =} \KeywordTok{col_character}\NormalTok{(),}
   \DataTypeTok{results =} \KeywordTok{col_integer}\NormalTok{(),}
   \DataTypeTok{num_clicks =} \KeywordTok{col_integer}\NormalTok{(),}
   \DataTypeTok{first_click =} \KeywordTok{col_integer}\NormalTok{()}
\NormalTok{ )}
\end{Highlighting}
\end{Shaded}

\begin{verbatim}
## cols(
##   session_id = col_character(),
##   search_index = col_integer(),
##   session_start_timestamp = col_double(),
##   session_end_timestamp = col_double(),
##   session_start_date = col_datetime(format = ""),
##   session_end_date = col_datetime(format = ""),
##   checkin = col_integer(),
##   group = col_character(),
##   results = col_integer(),
##   num_clicks = col_integer(),
##   first_click = col_integer()
## )
\end{verbatim}

\begin{itemize}
\tightlist
\item
  \textbf{first\_click:}
\end{itemize}

Inicialmente, foi necessário, a partir dos dos disponibilizados, gerar
novas variáveis.

Primeiramente, um refinamento dos dados foi realizado com o objetivo de
gerar novas variáveis que pudessem extrair algumas informações
valiosas. Tais variáveis, foram utilizadas como fonte de informação
para investigar as questões de pesquisa deste relatório.

O objeto principal da análise são as buscas e a navegação depois da
busca. Criamos esses dados a partir dos dados originais da wikimedia em
\texttt{/data/search\_data.csv}.

Aqui, exploramos esses dados.

\begin{Shaded}
\begin{Highlighting}[]
\NormalTok{buscas }\OperatorTok\StringTok{ }
\StringTok{    }\KeywordTok{ggplot}\NormalTok{(}\KeywordTok{aes}\NormalTok{(}\DataTypeTok{x =}\NormalTok{ results)) }\OperatorTok{+}\StringTok{ }
\StringTok{    }\KeywordTok{geom_histogram}\NormalTok{(}\DataTypeTok{binwidth =} \DecValTok{5}\NormalTok{) }
\end{Highlighting}
\end{Shaded}

\includegraphics{eda-searches_files/figure-latex/unnamed-chunk-1-1.pdf}


\end{document}
